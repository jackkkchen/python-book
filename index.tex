% Options for packages loaded elsewhere
\PassOptionsToPackage{unicode}{hyperref}
\PassOptionsToPackage{hyphens}{url}
\PassOptionsToPackage{dvipsnames,svgnames,x11names}{xcolor}
%
\documentclass[
  letterpaper,
  DIV=11,
  numbers=noendperiod]{scrreprt}

\usepackage{amsmath,amssymb}
\usepackage{iftex}
\ifPDFTeX
  \usepackage[T1]{fontenc}
  \usepackage[utf8]{inputenc}
  \usepackage{textcomp} % provide euro and other symbols
\else % if luatex or xetex
  \usepackage{unicode-math}
  \defaultfontfeatures{Scale=MatchLowercase}
  \defaultfontfeatures[\rmfamily]{Ligatures=TeX,Scale=1}
\fi
\usepackage{lmodern}
\ifPDFTeX\else  
    % xetex/luatex font selection
\fi
% Use upquote if available, for straight quotes in verbatim environments
\IfFileExists{upquote.sty}{\usepackage{upquote}}{}
\IfFileExists{microtype.sty}{% use microtype if available
  \usepackage[]{microtype}
  \UseMicrotypeSet[protrusion]{basicmath} % disable protrusion for tt fonts
}{}
\makeatletter
\@ifundefined{KOMAClassName}{% if non-KOMA class
  \IfFileExists{parskip.sty}{%
    \usepackage{parskip}
  }{% else
    \setlength{\parindent}{0pt}
    \setlength{\parskip}{6pt plus 2pt minus 1pt}}
}{% if KOMA class
  \KOMAoptions{parskip=half}}
\makeatother
\usepackage{xcolor}
\setlength{\emergencystretch}{3em} % prevent overfull lines
\setcounter{secnumdepth}{5}
% Make \paragraph and \subparagraph free-standing
\ifx\paragraph\undefined\else
  \let\oldparagraph\paragraph
  \renewcommand{\paragraph}[1]{\oldparagraph{#1}\mbox{}}
\fi
\ifx\subparagraph\undefined\else
  \let\oldsubparagraph\subparagraph
  \renewcommand{\subparagraph}[1]{\oldsubparagraph{#1}\mbox{}}
\fi

\usepackage{color}
\usepackage{fancyvrb}
\newcommand{\VerbBar}{|}
\newcommand{\VERB}{\Verb[commandchars=\\\{\}]}
\DefineVerbatimEnvironment{Highlighting}{Verbatim}{commandchars=\\\{\}}
% Add ',fontsize=\small' for more characters per line
\usepackage{framed}
\definecolor{shadecolor}{RGB}{241,243,245}
\newenvironment{Shaded}{\begin{snugshade}}{\end{snugshade}}
\newcommand{\AlertTok}[1]{\textcolor[rgb]{0.68,0.00,0.00}{#1}}
\newcommand{\AnnotationTok}[1]{\textcolor[rgb]{0.37,0.37,0.37}{#1}}
\newcommand{\AttributeTok}[1]{\textcolor[rgb]{0.40,0.45,0.13}{#1}}
\newcommand{\BaseNTok}[1]{\textcolor[rgb]{0.68,0.00,0.00}{#1}}
\newcommand{\BuiltInTok}[1]{\textcolor[rgb]{0.00,0.23,0.31}{#1}}
\newcommand{\CharTok}[1]{\textcolor[rgb]{0.13,0.47,0.30}{#1}}
\newcommand{\CommentTok}[1]{\textcolor[rgb]{0.37,0.37,0.37}{#1}}
\newcommand{\CommentVarTok}[1]{\textcolor[rgb]{0.37,0.37,0.37}{\textit{#1}}}
\newcommand{\ConstantTok}[1]{\textcolor[rgb]{0.56,0.35,0.01}{#1}}
\newcommand{\ControlFlowTok}[1]{\textcolor[rgb]{0.00,0.23,0.31}{#1}}
\newcommand{\DataTypeTok}[1]{\textcolor[rgb]{0.68,0.00,0.00}{#1}}
\newcommand{\DecValTok}[1]{\textcolor[rgb]{0.68,0.00,0.00}{#1}}
\newcommand{\DocumentationTok}[1]{\textcolor[rgb]{0.37,0.37,0.37}{\textit{#1}}}
\newcommand{\ErrorTok}[1]{\textcolor[rgb]{0.68,0.00,0.00}{#1}}
\newcommand{\ExtensionTok}[1]{\textcolor[rgb]{0.00,0.23,0.31}{#1}}
\newcommand{\FloatTok}[1]{\textcolor[rgb]{0.68,0.00,0.00}{#1}}
\newcommand{\FunctionTok}[1]{\textcolor[rgb]{0.28,0.35,0.67}{#1}}
\newcommand{\ImportTok}[1]{\textcolor[rgb]{0.00,0.46,0.62}{#1}}
\newcommand{\InformationTok}[1]{\textcolor[rgb]{0.37,0.37,0.37}{#1}}
\newcommand{\KeywordTok}[1]{\textcolor[rgb]{0.00,0.23,0.31}{#1}}
\newcommand{\NormalTok}[1]{\textcolor[rgb]{0.00,0.23,0.31}{#1}}
\newcommand{\OperatorTok}[1]{\textcolor[rgb]{0.37,0.37,0.37}{#1}}
\newcommand{\OtherTok}[1]{\textcolor[rgb]{0.00,0.23,0.31}{#1}}
\newcommand{\PreprocessorTok}[1]{\textcolor[rgb]{0.68,0.00,0.00}{#1}}
\newcommand{\RegionMarkerTok}[1]{\textcolor[rgb]{0.00,0.23,0.31}{#1}}
\newcommand{\SpecialCharTok}[1]{\textcolor[rgb]{0.37,0.37,0.37}{#1}}
\newcommand{\SpecialStringTok}[1]{\textcolor[rgb]{0.13,0.47,0.30}{#1}}
\newcommand{\StringTok}[1]{\textcolor[rgb]{0.13,0.47,0.30}{#1}}
\newcommand{\VariableTok}[1]{\textcolor[rgb]{0.07,0.07,0.07}{#1}}
\newcommand{\VerbatimStringTok}[1]{\textcolor[rgb]{0.13,0.47,0.30}{#1}}
\newcommand{\WarningTok}[1]{\textcolor[rgb]{0.37,0.37,0.37}{\textit{#1}}}

\providecommand{\tightlist}{%
  \setlength{\itemsep}{0pt}\setlength{\parskip}{0pt}}\usepackage{longtable,booktabs,array}
\usepackage{calc} % for calculating minipage widths
% Correct order of tables after \paragraph or \subparagraph
\usepackage{etoolbox}
\makeatletter
\patchcmd\longtable{\par}{\if@noskipsec\mbox{}\fi\par}{}{}
\makeatother
% Allow footnotes in longtable head/foot
\IfFileExists{footnotehyper.sty}{\usepackage{footnotehyper}}{\usepackage{footnote}}
\makesavenoteenv{longtable}
\usepackage{graphicx}
\makeatletter
\def\maxwidth{\ifdim\Gin@nat@width>\linewidth\linewidth\else\Gin@nat@width\fi}
\def\maxheight{\ifdim\Gin@nat@height>\textheight\textheight\else\Gin@nat@height\fi}
\makeatother
% Scale images if necessary, so that they will not overflow the page
% margins by default, and it is still possible to overwrite the defaults
% using explicit options in \includegraphics[width, height, ...]{}
\setkeys{Gin}{width=\maxwidth,height=\maxheight,keepaspectratio}
% Set default figure placement to htbp
\makeatletter
\def\fps@figure{htbp}
\makeatother

\KOMAoption{captions}{tableheading}
\makeatletter
\makeatother
\makeatletter
\@ifpackageloaded{bookmark}{}{\usepackage{bookmark}}
\makeatother
\makeatletter
\@ifpackageloaded{caption}{}{\usepackage{caption}}
\AtBeginDocument{%
\ifdefined\contentsname
  \renewcommand*\contentsname{Table of contents}
\else
  \newcommand\contentsname{Table of contents}
\fi
\ifdefined\listfigurename
  \renewcommand*\listfigurename{List of Figures}
\else
  \newcommand\listfigurename{List of Figures}
\fi
\ifdefined\listtablename
  \renewcommand*\listtablename{List of Tables}
\else
  \newcommand\listtablename{List of Tables}
\fi
\ifdefined\figurename
  \renewcommand*\figurename{Figure}
\else
  \newcommand\figurename{Figure}
\fi
\ifdefined\tablename
  \renewcommand*\tablename{Table}
\else
  \newcommand\tablename{Table}
\fi
}
\@ifpackageloaded{float}{}{\usepackage{float}}
\floatstyle{ruled}
\@ifundefined{c@chapter}{\newfloat{codelisting}{h}{lop}}{\newfloat{codelisting}{h}{lop}[chapter]}
\floatname{codelisting}{Listing}
\newcommand*\listoflistings{\listof{codelisting}{List of Listings}}
\makeatother
\makeatletter
\@ifpackageloaded{caption}{}{\usepackage{caption}}
\@ifpackageloaded{subcaption}{}{\usepackage{subcaption}}
\makeatother
\makeatletter
\@ifpackageloaded{tcolorbox}{}{\usepackage[skins,breakable]{tcolorbox}}
\makeatother
\makeatletter
\@ifundefined{shadecolor}{\definecolor{shadecolor}{rgb}{.97, .97, .97}}
\makeatother
\makeatletter
\makeatother
\makeatletter
\makeatother
\ifLuaTeX
  \usepackage{selnolig}  % disable illegal ligatures
\fi
\IfFileExists{bookmark.sty}{\usepackage{bookmark}}{\usepackage{hyperref}}
\IfFileExists{xurl.sty}{\usepackage{xurl}}{} % add URL line breaks if available
\urlstyle{same} % disable monospaced font for URLs
\hypersetup{
  pdftitle={MySQL入门学习},
  pdfauthor={明数科技},
  colorlinks=true,
  linkcolor={blue},
  filecolor={Maroon},
  citecolor={Blue},
  urlcolor={Blue},
  pdfcreator={LaTeX via pandoc}}

\title{MySQL入门学习}
\author{明数科技}
\date{2023-05-25}

\begin{document}
\maketitle
\ifdefined\Shaded\renewenvironment{Shaded}{\begin{tcolorbox}[breakable, sharp corners, enhanced, frame hidden, boxrule=0pt, borderline west={3pt}{0pt}{shadecolor}, interior hidden]}{\end{tcolorbox}}\fi

\renewcommand*\contentsname{Table of contents}
{
\hypersetup{linkcolor=}
\setcounter{tocdepth}{2}
\tableofcontents
}
\bookmarksetup{startatroot}

\hypertarget{ux524dux8a00}{%
\chapter*{前言}\label{ux524dux8a00}}
\addcontentsline{toc}{chapter}{前言}

\markboth{前言}{前言}

这是由明数科技归纳整理出的 \textbf{MySQL} 入门教程,专门针对新手小白~

\bookmarksetup{startatroot}

\hypertarget{mysqlux4ecbux7ecd}{%
\chapter{MySQL介绍}\label{mysqlux4ecbux7ecd}}

\hypertarget{mysqlux4e09ux5c42ux7ed3ux6784}{%
\section{MySQL三层结构}\label{mysqlux4e09ux5c42ux7ed3ux6784}}

\hypertarget{ux6570ux636eux5e93}{%
\subsection{数据库}\label{ux6570ux636eux5e93}}

全称是\textbf{存储数据的仓库},数据时候有组织的进行存储;简称DataBase(DB)

\hypertarget{ux6570ux636eux5e93ux7ba1ux7406ux7cfbux7edf}{%
\subsection{数据库管理系统}\label{ux6570ux636eux5e93ux7ba1ux7406ux7cfbux7edf}}

全称是\textbf{操纵和管理数据库的大型软件} ;简称DataBase Management
System(DBMS)

\hypertarget{sql}{%
\subsection{SQL}\label{sql}}

操纵\textbf{关系型数据库的编程语言},定义了一套操作关系型数据库\textbf{统一标准}
;简称Structured Query Language(SQL)

\hypertarget{ux4e3bux6d41ux7684ux5173ux7cfbux578bux6570ux636eux5e93ux7ba1ux7406ux7cfbux7edfux7cfbux7edf}{%
\section{主流的关系型数据库管理系统系统}\label{ux4e3bux6d41ux7684ux5173ux7cfbux578bux6570ux636eux5e93ux7ba1ux7406ux7cfbux7edfux7cfbux7edf}}

● Oracle

● MySQL

● Microsoft SQL Server

\bookmarksetup{startatroot}

\hypertarget{sqlux5b66ux4e60}{%
\chapter{SQL学习}\label{sqlux5b66ux4e60}}

\hypertarget{sqlux901aux7528ux8bedux6cd5}{%
\section{SQL通用语法}\label{sqlux901aux7528ux8bedux6cd5}}

SQL语句可以单行或多行书写 以\textbf{分号结尾}

SQL语句可以使用空格/缩进来增强语句的可读性

MySQL数据库的SQL语句\textbf{不区分大小写,关键字建议使用大写}

● 注释

\begin{enumerate}
\def\labelenumi{\arabic{enumi}.}
\tightlist
\item
  单行注释 :-- 注释内容 或\#注释内容(MySQL独有)
\item
  多行注释 :/* */
\end{enumerate}

\hypertarget{sqlux5206ux7c7b}{%
\section{SQL分类}\label{sqlux5206ux7c7b}}

● SQL语句,根据其功能,主要分为四类:DDL、DML、DQL、DCL。

\hypertarget{ddl-ux6570ux636eux5b9aux4e49ux8bedux53e5}{%
\subsection{\texorpdfstring{DDL
\textbf{数据定义语句}}{DDL 数据定义语句}}\label{ddl-ux6570ux636eux5b9aux4e49ux8bedux53e5}}

\hypertarget{ux5e93ux64cdux4f5c}{%
\subsubsection{库操作}\label{ux5e93ux64cdux4f5c}}

● 查询

1)查询所有数据库

\begin{Shaded}
\begin{Highlighting}[]
\NormalTok{(SHOW)SHOW DATABASES;}
\end{Highlighting}
\end{Shaded}

2)查询当前数据库

\begin{Shaded}
\begin{Highlighting}[]
\NormalTok{(}\KeywordTok{SELECT}\NormalTok{)}\KeywordTok{SELECT} \KeywordTok{DATABASE}\NormalTok{();}
\end{Highlighting}
\end{Shaded}

● 创建(CREATE)

\begin{Shaded}
\begin{Highlighting}[]
\KeywordTok{CREATE} \KeywordTok{DATABASE}\NormalTok{[}\ControlFlowTok{IF} \KeywordTok{NOT} \KeywordTok{EXISTS}\NormalTok{]数据库名[}\KeywordTok{DEFAULT}\NormalTok{ CHARSET 字符集][COLLATE 排序规则];}
\end{Highlighting}
\end{Shaded}

● 删除(DROP)

\begin{Shaded}
\begin{Highlighting}[]
\KeywordTok{DROP}\NormalTok{ DATABASES[}\ControlFlowTok{IF} \KeywordTok{EXISTS}\NormalTok{] 数据库名;}
\end{Highlighting}
\end{Shaded}

● 使用(USE)/\emph{切换访问的数据库}/

\begin{Shaded}
\begin{Highlighting}[]
\NormalTok{USE数据库名;}
\end{Highlighting}
\end{Shaded}

DDL 数据定义语句

\hypertarget{ux8868ux64cdux4f5c}{%
\subsubsection{表操作}\label{ux8868ux64cdux4f5c}}

● 查询

1)查询当前数据所有表

\begin{Shaded}
\begin{Highlighting}[]
\NormalTok{SHOW }\KeywordTok{TABLES}\NormalTok{;}
\end{Highlighting}
\end{Shaded}

2)查询表结构

\begin{Shaded}
\begin{Highlighting}[]
\KeywordTok{DESC}\NormalTok{ 表名;}
\end{Highlighting}
\end{Shaded}

3)查询指定表的建表语句

\begin{Shaded}
\begin{Highlighting}[]
\NormalTok{SHOW }\KeywordTok{CREATE} \KeywordTok{TABLE}\NormalTok{ 表名;}
\end{Highlighting}
\end{Shaded}

● 创建(CREATE)

\begin{Shaded}
\begin{Highlighting}[]
\KeywordTok{CREATE} \KeywordTok{TABLE}\NormalTok{ 表名\{}

\NormalTok{字段1 字段1类型[}\KeywordTok{COMMENT}\NormalTok{ 字段1注释],}

\NormalTok{字段2 字段2类型[}\KeywordTok{COMMENT}\NormalTok{ 字段2注释],}

\NormalTok{字段3 字段3类型[}\KeywordTok{COMMENT}\NormalTok{ 字段3注释]}

\NormalTok{\}[}\KeywordTok{COMMENT}\NormalTok{ 表注释];}
\end{Highlighting}
\end{Shaded}

● 修改(ALTER)

\begin{enumerate}
\def\labelenumi{\arabic{enumi}.}
\tightlist
\item
  添加字段(ADD)
\end{enumerate}

\begin{Shaded}
\begin{Highlighting}[]
\KeywordTok{ALTER} \KeywordTok{TABLE}\NormalTok{ 表名 }\KeywordTok{ADD}\NormalTok{ 字段名 类型(长度)[}\KeywordTok{comment}\NormalTok{ 注释][约束];}
\end{Highlighting}
\end{Shaded}

\begin{enumerate}
\def\labelenumi{\arabic{enumi}.}
\setcounter{enumi}{1}
\tightlist
\item
  修改数据类型(MODIFY)
\end{enumerate}

\begin{Shaded}
\begin{Highlighting}[]
\KeywordTok{ALTER} \KeywordTok{TABLE}\NormalTok{ 表名 }\KeywordTok{MODIFY}\NormalTok{ 字段名 新数据类型(长度);}
\end{Highlighting}
\end{Shaded}

\begin{enumerate}
\def\labelenumi{\arabic{enumi}.}
\setcounter{enumi}{2}
\tightlist
\item
  修改字段名和字段类型(CHANGE)
\end{enumerate}

\begin{Shaded}
\begin{Highlighting}[]
\KeywordTok{ALTER} \KeywordTok{TABLE}\NormalTok{ 表名 }\KeywordTok{CHANGE}\NormalTok{ 旧字段名 新字段名 类型(长度)[}\KeywordTok{COMMENT}\NormalTok{ 注释] [约束];}
\end{Highlighting}
\end{Shaded}

\begin{enumerate}
\def\labelenumi{\arabic{enumi}.}
\setcounter{enumi}{3}
\tightlist
\item
  删除字段(DROP)
\end{enumerate}

\begin{Shaded}
\begin{Highlighting}[]
\KeywordTok{ALTER} \KeywordTok{TABLE}\NormalTok{ 表名 }\KeywordTok{DROP}\NormalTok{ 字段名;\#删除列}
\end{Highlighting}
\end{Shaded}

\begin{enumerate}
\def\labelenumi{\arabic{enumi}.}
\setcounter{enumi}{4}
\tightlist
\item
  修改表名
\end{enumerate}

\begin{Shaded}
\begin{Highlighting}[]
\KeywordTok{ALTER} \KeywordTok{TABLE}\NormalTok{ 表名 }\KeywordTok{RENAME} \KeywordTok{TO}\NormalTok{ 新表名;}
\end{Highlighting}
\end{Shaded}

● 删除

\begin{enumerate}
\def\labelenumi{\arabic{enumi}.}
\tightlist
\item
  删除表
\end{enumerate}

\begin{Shaded}
\begin{Highlighting}[]
\KeywordTok{DROP} \KeywordTok{TABLE}\NormalTok{ [}\ControlFlowTok{IF} \KeywordTok{EXISTS}\NormalTok{]表名;}
\end{Highlighting}
\end{Shaded}

\begin{enumerate}
\def\labelenumi{\arabic{enumi}.}
\setcounter{enumi}{1}
\tightlist
\item
  删除指定表,并重新创建该表(该表的\textbf{数据会被删除},以该名表被重新创建)
\end{enumerate}

\begin{Shaded}
\begin{Highlighting}[]
\KeywordTok{TRUNCATE} \KeywordTok{TABLE}\NormalTok{ 表名;}
\end{Highlighting}
\end{Shaded}

\hypertarget{dml-ux6570ux636eux64cdux4f5cux8bedux53e5}{%
\subsubsection{DML
数据操作语句}\label{dml-ux6570ux636eux64cdux4f5cux8bedux53e5}}

\hypertarget{ux6dfbux52a0ux6570ux636einsert}{%
\paragraph{● 添加数据(INSERT)}\label{ux6dfbux52a0ux6570ux636einsert}}

\begin{enumerate}
\def\labelenumi{\arabic{enumi}.}
\tightlist
\item
  给\textbf{指定字段}名添加数据
\end{enumerate}

\begin{Shaded}
\begin{Highlighting}[]
\KeywordTok{INSERT} \KeywordTok{INTO}\NormalTok{ 表名  (字段名1,字段名2, . . .)}\KeywordTok{VALUES}\NormalTok{(值1,值2, . . .);}
\end{Highlighting}
\end{Shaded}

\begin{enumerate}
\def\labelenumi{\arabic{enumi}.}
\setcounter{enumi}{1}
\tightlist
\item
  给\textbf{全部字段名}添加数据
\end{enumerate}

\begin{Shaded}
\begin{Highlighting}[]
\KeywordTok{INSERT} \KeywordTok{INTO}\NormalTok{ 表名 }\KeywordTok{VALUES}\NormalTok{(值1,值2, . . .);}
\end{Highlighting}
\end{Shaded}

\begin{enumerate}
\def\labelenumi{\arabic{enumi}.}
\setcounter{enumi}{2}
\tightlist
\item
  批量添加数据
\end{enumerate}

/\emph{批量插入数据 用逗号隔开}/

\begin{Shaded}
\begin{Highlighting}[]
\KeywordTok{INSERT} \KeywordTok{INTO}\NormalTok{ 表名(字段名1,字段名2, . . .)}\KeywordTok{VALUES}\NormalTok{(值1,值2, . . .),(值1,值2, . . .),(值 1,值2, . . .);}

\KeywordTok{INSERT} \KeywordTok{INTO}\NormalTok{ 表名 }\KeywordTok{VALUES}\NormalTok{ (值1,值2, . . .), (值1,值2, . . . .), (值1,值2, . . . .);}
\end{Highlighting}
\end{Shaded}

\textbf{注意:}

1)插入数据时,指定的字段顺序需要与值的\textbf{顺序是一致}的

2)字符串和日期类型应该包含的在引号中

3)插入的数据大小,应该在字段的规定范围内

\hypertarget{ux4feeux6539ux6570ux636eupdate}{%
\paragraph{● 修改数据(UPDATE)}\label{ux4feeux6539ux6570ux636eupdate}}

\begin{Shaded}
\begin{Highlighting}[]
\KeywordTok{UPDATE}\NormalTok{ 表名 }\KeywordTok{SET}\NormalTok{ 字段名1}\OperatorTok{=}\NormalTok{值1,字段名2}\OperatorTok{=}\NormalTok{值2, . . .[}\KeywordTok{WHERE}\NormalTok{ 条件];}
\end{Highlighting}
\end{Shaded}

\hypertarget{ux5220ux9664delete}{%
\paragraph{● 删除(DELETE)}\label{ux5220ux9664delete}}

\begin{Shaded}
\begin{Highlighting}[]
\KeywordTok{DELETE} \KeywordTok{FROM}\NormalTok{ 表名[}\KeywordTok{WHERE}\NormalTok{ 条件];}
\end{Highlighting}
\end{Shaded}

\hypertarget{dql-ux6570ux636eux67e5ux8be2ux8bedux53e5}{%
\subsubsection{DQL
数据查询语句}\label{dql-ux6570ux636eux67e5ux8be2ux8bedux53e5}}

\hypertarget{ux57faux672cux67e5ux8be2}{%
\paragraph{基本查询}\label{ux57faux672cux67e5ux8be2}}

\begin{enumerate}
\def\labelenumi{\arabic{enumi}.}
\tightlist
\item
  查询多个字段
\end{enumerate}

\begin{Shaded}
\begin{Highlighting}[]
\KeywordTok{SELECT}\NormalTok{ 字段1,字段2,字段3 . . .}\KeywordTok{FROM}\NormalTok{ 表名;}

\KeywordTok{SELECT} \OperatorTok{*}\KeywordTok{FROM}\NormalTok{ 表名(全部字段)\#返回查询列表中所有数据}
\end{Highlighting}
\end{Shaded}

\begin{enumerate}
\def\labelenumi{\arabic{enumi}.}
\setcounter{enumi}{1}
\tightlist
\item
  设置别名(AS)
\end{enumerate}

\begin{Shaded}
\begin{Highlighting}[]
\KeywordTok{SELECT}\NormalTok{ 字段1[}\KeywordTok{AS}\NormalTok{ 别名1],字段2[}\KeywordTok{AS}\NormalTok{ 别名2]. . .FROM表名;\#别名要用单引号 AS可以省略}
\end{Highlighting}
\end{Shaded}

\begin{enumerate}
\def\labelenumi{\arabic{enumi}.}
\setcounter{enumi}{2}
\tightlist
\item
  去除重复纪录(DISTINCT)
\end{enumerate}

\begin{Shaded}
\begin{Highlighting}[]
\KeywordTok{SELECT} \KeywordTok{DISTINCT}\NormalTok{ 字段列表 }\KeywordTok{FROM}\NormalTok{ 表名;\#将该列的重复的元素去除(去重)}
\end{Highlighting}
\end{Shaded}

\hypertarget{ux6761ux4ef6ux67e5ux8be2}{%
\paragraph{2.条件查询}\label{ux6761ux4ef6ux67e5ux8be2}}

\begin{enumerate}
\def\labelenumi{\arabic{enumi}.}
\tightlist
\item
  语法(WHERE)
\end{enumerate}

\begin{Shaded}
\begin{Highlighting}[]
\KeywordTok{SELECT}\NormalTok{ 字段列表 }\KeywordTok{FROM} \KeywordTok{WHERE}\NormalTok{ 条件列表; \# where相当于java中if语句 后接条件语句 字段列表 }\OperatorTok{=*}\NormalTok{(所有列)}
\end{Highlighting}
\end{Shaded}

\begin{enumerate}
\def\labelenumi{\arabic{enumi}.}
\setcounter{enumi}{1}
\tightlist
\item
  条件
\end{enumerate}

\hspace{0pt} ● 比较运算符

\begin{longtable}[]{@{}
  >{\raggedright\arraybackslash}p{(\columnwidth - 2\tabcolsep) * \real{0.2029}}
  >{\raggedright\arraybackslash}p{(\columnwidth - 2\tabcolsep) * \real{0.7971}}@{}}
\toprule\noalign{}
\begin{minipage}[b]{\linewidth}\raggedright
\textbf{比较运算符}
\end{minipage} & \begin{minipage}[b]{\linewidth}\raggedright
\textbf{功能}
\end{minipage} \\
\midrule\noalign{}
\endhead
\bottomrule\noalign{}
\endlastfoot
\textgreater{} & 大于 \\
\textgreater= & 大于等于 \\
\textless{} & 小于 \\
\textless= & 小于等于 \\
= & 等于 \\
\textless\textgreater 或! = & 不等于 \\
BETWEEN\ldots AND & 在某个范围之内(含最小、最大值) \\
IN(\ldots) & 在in之后的列表中的值,多选一 \\
LIKE 占位符 & 模糊匹配 ( \_ 匹配\textbf{单个字符}, \%
匹配\textbf{任意个字符 }) \\
IS NULL & 是NULL \\
\end{longtable}

\hspace{0pt} ● 逻辑运算符

\begin{longtable}[]{@{}ll@{}}
\toprule\noalign{}
\textbf{逻辑运算符} & \textbf{功能} \\
\midrule\noalign{}
\endhead
\bottomrule\noalign{}
\endlastfoot
AND或\&\& & 并且(多个条件同时成立) \\
OR或\textbar\textbar{} & 或者(多个条件任意一个成立) \\
NOT或! & 非,不是 \\
\end{longtable}

\hspace{0pt} ● 聚合函数

\begin{enumerate}
\def\labelenumi{\arabic{enumi}.}
\tightlist
\item
  介绍
\end{enumerate}

将一列数据作为一个整体

\begin{enumerate}
\def\labelenumi{\arabic{enumi}.}
\setcounter{enumi}{1}
\tightlist
\item
  常见聚合函数
\end{enumerate}

\begin{longtable}[]{@{}ll@{}}
\toprule\noalign{}
\textbf{函数} & \textbf{功能} \\
\midrule\noalign{}
\endhead
\bottomrule\noalign{}
\endlastfoot
count & 统计数量 \\
max & 最大值 \\
min & 最小值 \\
avg & 平均值 \\
sum & 求和 \\
\end{longtable}

\begin{Shaded}
\begin{Highlighting}[]
\KeywordTok{SELECT}\NormalTok{ 聚合函数 }\KeywordTok{from}\NormalTok{ 表名;}

\KeywordTok{SELECT} \FunctionTok{COUNT}\NormalTok{(}\OperatorTok{*}\NormalTok{)}\KeywordTok{FROM}\NormalTok{ EMP;}

\KeywordTok{SELECT} \FunctionTok{COUNT}\NormalTok{(}\KeywordTok{ID}\NormalTok{)}\KeywordTok{FROM}\NormalTok{ EMP;}

\NormalTok{\#聚合函数传进去的是字段}
\end{Highlighting}
\end{Shaded}

注意:null值不参与聚合函数的运算

\hypertarget{ux5206ux7ec4ux67e5ux8be2group-by}{%
\paragraph{3.分组查询(GROUP
BY)}\label{ux5206ux7ec4ux67e5ux8be2group-by}}

\begin{enumerate}
\def\labelenumi{\arabic{enumi}.}
\tightlist
\item
  语法
\end{enumerate}

\begin{Shaded}
\begin{Highlighting}[]
\KeywordTok{SELECT}\NormalTok{ 字段列表 }\KeywordTok{FROM}\NormalTok{ 表名[}\KeywordTok{WHERE}\NormalTok{ 条件]}\KeywordTok{GROUP} \KeywordTok{BY}\NormalTok{ 分组字段名[}\KeywordTok{HAVING}\NormalTok{ 分组后过滤条件];}

\KeywordTok{SELECT}\NormalTok{ gender,}\FunctionTok{count}\NormalTok{(}\OperatorTok{*}\NormalTok{)}\KeywordTok{from}\NormalTok{ emp }\KeywordTok{GROUP} \KeywordTok{BY}\NormalTok{ gender;}

\NormalTok{\#查询效果是 根据性别分组 统计男女的数量}

\KeywordTok{SELECT}\NormalTok{ gender,}\FunctionTok{AVG}\NormalTok{(age)}\KeywordTok{FROM}\NormalTok{ emp }\KeywordTok{GROUP} \KeywordTok{BY}\NormalTok{ gender;}

\NormalTok{\#查询效果是 根据性别分组 统计男女的平均年龄}

\KeywordTok{SELECT}\NormalTok{ address,}\FunctionTok{COUNT}\NormalTok{(}\OperatorTok{*}\NormalTok{)}\KeywordTok{FROM}\NormalTok{ emp }\KeywordTok{WHERE}\NormalTok{ age}\OperatorTok{\textless{}}\NormalTok{xxx }\KeywordTok{GROUP} \KeywordTok{BY}\NormalTok{ address;}

\NormalTok{\#查询效果是 根据地址分组 统计年龄小于xxx的人数}

\KeywordTok{SELECT}\NormalTok{ address,}\FunctionTok{COUNT}\NormalTok{(}\OperatorTok{*}\NormalTok{) address\_count }\KeywordTok{FROM}\NormalTok{ emp }\KeywordTok{WHERE}\NormalTok{ age}\OperatorTok{\textless{}}\NormalTok{xxx }\KeywordTok{GROUP} \KeywordTok{BY}\NormalTok{ address }\KeywordTok{HAVING}\NormalTok{ address\_count}\OperatorTok{\textgreater{}}\NormalTok{X;}

\NormalTok{\#上面一条语句的基础上再次筛选having}
\end{Highlighting}
\end{Shaded}

\begin{enumerate}
\def\labelenumi{\arabic{enumi}.}
\setcounter{enumi}{1}
\tightlist
\item
  HAING 和WHERE区别
\end{enumerate}

执行时机不同:

where是\textbf{分组之前}进行过滤,不满足where条件不进行分组;而having是
\textbf{分组之后}对结果进行过滤

判断条件不同:

\textbf{where不能对聚合函数进行判断} 而having可以。

\begin{enumerate}
\def\labelenumi{\arabic{enumi}.}
\setcounter{enumi}{2}
\tightlist
\item
  注意
\end{enumerate}

执行顺序:where-\textgreater 聚合函数-\textgreater having

分组之后,查询的字段一般为聚合函数和分组字段,查询其他字段无任何意义(意思是分组字段和查询字段是相同的)

\hypertarget{ux6392ux5e8fux67e5ux8be2order-by}{%
\paragraph{4.排序查询(ORDER
BY)}\label{ux6392ux5e8fux67e5ux8be2order-by}}

\begin{enumerate}
\def\labelenumi{\arabic{enumi}.}
\tightlist
\item
  语法
\end{enumerate}

\begin{Shaded}
\begin{Highlighting}[]
\NormalTok{\#支持多字段排序}
\KeywordTok{SELECT}\NormalTok{ 字段列表 }\KeywordTok{FROM}\NormalTok{ 表名 }\KeywordTok{ORDER} \KeywordTok{BY}\NormalTok{ 字段1 排序方1,字段2 排序方式2 }\KeywordTok{SELECT} \OperatorTok{*} \KeywordTok{FROM}\NormalTok{ emp }\KeywordTok{ORDER} \KeywordTok{BY}\NormalTok{ age }\KeywordTok{desc}\NormalTok{;\#降序}

\KeywordTok{SELECT} \OperatorTok{*} \KeywordTok{FROM}\NormalTok{ emp }\KeywordTok{ORDER} \KeywordTok{BY}\NormalTok{ age }\KeywordTok{asc}\NormalTok{;\#升序}

\KeywordTok{SELECT} \OperatorTok{*} \KeywordTok{FROM}\NormalTok{ emp }\KeywordTok{ORDER} \KeywordTok{BY}\NormalTok{ age }\KeywordTok{asc}\NormalTok{,entrydate }\KeywordTok{desc}\NormalTok{;}
\end{Highlighting}
\end{Shaded}

\begin{enumerate}
\def\labelenumi{\arabic{enumi}.}
\setcounter{enumi}{1}
\tightlist
\item
  排序方式
\end{enumerate}

ASC:升序(默认值)

DESC:降序

注意:如果是多字段查询时,当第一个字段值相同时,才会根据第二个字段进行排序

\hypertarget{ux5206ux9875ux67e5ux8be2limit}{%
\paragraph{5.分页查询(LIMIT)}\label{ux5206ux9875ux67e5ux8be2limit}}

\begin{enumerate}
\def\labelenumi{\arabic{enumi}.}
\tightlist
\item
  语法
\end{enumerate}

\begin{Shaded}
\begin{Highlighting}[]
\KeywordTok{SELECT}\NormalTok{ 字段列表 }\KeywordTok{FROM} \KeywordTok{LIMIT}\NormalTok{ 起始索引 ,查询记录数;\#两个参数}
\end{Highlighting}
\end{Shaded}

注意

1)起始索引是\textbf{从0开始,}起始索引 =(查询页码-1)*每页显示记录数

2)\textbf{分页查询是数据库的方言},不同的数据库有不同的实现,
Mysql是LIMIT

3)如果查询的是第一页数据,起始索引可以省略,直接简写为limit 10

\hypertarget{ux51fdux6570}{%
\section{函数}\label{ux51fdux6570}}

\hypertarget{ux5b57ux7b26ux4e32ux51fdux6570}{%
\subsection{字符串函数}\label{ux5b57ux7b26ux4e32ux51fdux6570}}

\hspace{0pt} ● 常用函数

\begin{longtable}[]{@{}
  >{\raggedright\arraybackslash}p{(\columnwidth - 2\tabcolsep) * \real{0.2857}}
  >{\raggedright\arraybackslash}p{(\columnwidth - 2\tabcolsep) * \real{0.7143}}@{}}
\toprule\noalign{}
\begin{minipage}[b]{\linewidth}\raggedright
\textbf{函数}
\end{minipage} & \begin{minipage}[b]{\linewidth}\raggedright
\textbf{功能}
\end{minipage} \\
\midrule\noalign{}
\endhead
\bottomrule\noalign{}
\endlastfoot
CONCAT(S1,S2,\ldots Sn) & 字符串拼接,将s1,s2,..sn拼接成一个字符串 \\
LOWER(str) & 将字符串str全部转成小写 \\
UPPER(str) & 将字符串str全部转成大写 \\
LPAD(str,n,pad) &
左填充,用字符串pad对str的左边进行填充,达到n个字符串长度 \\
RPAD(str,n,pad) &
右填充,用字符串pad对str的右边进行填充,达到n个字符串长度 \\
TRIM(str) & 去掉字符串\textbf{头部和尾部的空格} \\
SUBSTRING(str,start,len) &
返回从字符串str从start位置起len个长度的字符串SELECT 函数(参数); \\
\end{longtable}

\hypertarget{ux6570ux503cux51fdux6570}{%
\subsection{2. 数值函数}\label{ux6570ux503cux51fdux6570}}

\hspace{0pt} ● 常用函数

\begin{longtable}[]{@{}ll@{}}
\toprule\noalign{}
\textbf{函数} & 功能 \\
\midrule\noalign{}
\endhead
\bottomrule\noalign{}
\endlastfoot
CEIL(x) & 向上取整 \\
FLOOR(x) & 向下取整 \\
MOD(x,y) & 返回x/y的模 \\
RAND() & 返回0\textasciitilde1的随机数 \\
ROUND(x,y) & 求参数x的四舍五入的值,保留y位小数 \\
\end{longtable}

\hypertarget{ux65e5ux671fux51fdux6570}{%
\subsection{\texorpdfstring{\textbf{3.}
\textbf{日期函数}}{3. 日期函数}}\label{ux65e5ux671fux51fdux6570}}

\hspace{0pt} ● 常用函数

\begin{longtable}[]{@{}
  >{\raggedright\arraybackslash}p{(\columnwidth - 2\tabcolsep) * \real{0.3929}}
  >{\raggedright\arraybackslash}p{(\columnwidth - 2\tabcolsep) * \real{0.6071}}@{}}
\toprule\noalign{}
\begin{minipage}[b]{\linewidth}\raggedright
\textbf{函数}
\end{minipage} & \begin{minipage}[b]{\linewidth}\raggedright
\textbf{功能}
\end{minipage} \\
\midrule\noalign{}
\endhead
\bottomrule\noalign{}
\endlastfoot
CURDATE() & 返回当前日期 \\
CURTIME() & 返回当前时间 \\
NOW() & 返回当前日期和时间 \\
YEAR(date) & 返回当前指定date的年份 \\
MONTH(date) & 返回当前指定date的月份 \\
DAY(date) & 返回当前指定date的日期 \\
DATE\_ADD(date,INTERVAL\_EXPR type) &
返回一个日期/时间加上一个\textbf{时间间隔}expr后的时间值 \\
DATEDIFF(date1 ,date2) & 返回起始时间date1和结束时间date2之间的天数 \\
\end{longtable}

举例:

\begin{Shaded}
\begin{Highlighting}[]
\KeywordTok{SELECT}\NormalTok{ date\_add(now(),}\DataTypeTok{INTERVAL} \DecValTok{70} \DataTypeTok{YEAR}\NormalTok{);}

\NormalTok{\#查询员工的入职天数}

\KeywordTok{SELECT}\NormalTok{ name,diff(now(),entrydate) }\KeywordTok{as}\NormalTok{ entrydayscount }\KeywordTok{FROM}\NormalTok{ emp }\KeywordTok{Group} \KeywordTok{By}\NormalTok{ entrydayscount }\KeywordTok{SELECT}\NormalTok{ name,DATEDIFF(NOW(),}\StringTok{\textquotesingle{}2000{-}11{-}12 \textquotesingle{}}\NormalTok{)}\KeywordTok{FROM}\NormalTok{ emp}
\end{Highlighting}
\end{Shaded}

\hypertarget{ux6d41ux7a0bux51fdux6570}{%
\subsection{4. 流程函数}\label{ux6d41ux7a0bux51fdux6570}}

\hspace{0pt} ● 常见函数

\begin{longtable}[]{@{}
  >{\raggedright\arraybackslash}p{(\columnwidth - 2\tabcolsep) * \real{0.4495}}
  >{\raggedright\arraybackslash}p{(\columnwidth - 2\tabcolsep) * \real{0.5505}}@{}}
\toprule\noalign{}
\begin{minipage}[b]{\linewidth}\raggedright
\textbf{函数}
\end{minipage} & \begin{minipage}[b]{\linewidth}\raggedright
功能
\end{minipage} \\
\midrule\noalign{}
\endhead
\bottomrule\noalign{}
\endlastfoot
IF(Value ,t ,f) & 如果value为true,则返回t,否则返回f \\
IFNULL(Value1,Value2) &
如果Value1不为空(空=null),则返回value1,否则返回value2 \\
CASE WHEN{[}val1{]}THEN{[}res1{]}\ldots ELSE{[}default{]}END &
如果val1为true,返回res1,否则返回default默认值 \\
CASE {[}expr{]}WHEN{[}val{]}THEN{[}res1{]}\ldots ELSE{[}default{]}END &
如果\textbf{expr(表达式)}的值等于val1,返回res1
,\ldots 否则返回default默认值 \\
\end{longtable}

举例:

\begin{Shaded}
\begin{Highlighting}[]
\KeywordTok{SELECT}\NormalTok{ name,(}\ControlFlowTok{case}\NormalTok{ workaddress }\ControlFlowTok{when} \StringTok{\textquotesingle{}北京 \textquotesingle{}}\NormalTok{then}\StringTok{\textquotesingle{}一线城市 \textquotesingle{}}\NormalTok{when }\StringTok{\textquotesingle{}上海 \textquotesingle{}}\NormalTok{then}\StringTok{\textquotesingle{}一线城 市 \textquotesingle{}}\NormalTok{else}\StringTok{\textquotesingle{}二线城市 \textquotesingle{}}\NormalTok{end) }\KeywordTok{as} \StringTok{\textquotesingle{}工作地址 \textquotesingle{}}

\KeywordTok{SELECT} \KeywordTok{id}\NormalTok{,name,(}\ControlFlowTok{case} \ControlFlowTok{when}\NormalTok{ math}\OperatorTok{\textgreater{}=}\DecValTok{85} \ControlFlowTok{then} \StringTok{\textquotesingle{}优秀 \textquotesingle{}} \ControlFlowTok{when}\NormalTok{ math}\OperatorTok{\textgreater{}=}\DecValTok{60} \ControlFlowTok{then} \StringTok{\textquotesingle{}及格 \textquotesingle{}}\NormalTok{else }\StringTok{\textquotesingle{}不 及格 \textquotesingle{}}\NormalTok{end)}\KeywordTok{as} \StringTok{\textquotesingle{}数学 \textquotesingle{}}\NormalTok{),}

\NormalTok{(}\ControlFlowTok{case} \ControlFlowTok{when}\NormalTok{ English}\OperatorTok{\textgreater{}=}\DecValTok{85} \ControlFlowTok{then} \StringTok{\textquotesingle{}优秀 \textquotesingle{}} \ControlFlowTok{when}\NormalTok{ English}\OperatorTok{\textgreater{}=}\DecValTok{60} \ControlFlowTok{then} \StringTok{\textquotesingle{}及格 \textquotesingle{}}\NormalTok{else }\StringTok{\textquotesingle{}不及格 \textquotesingle{}}\NormalTok{end)}\KeywordTok{as} \StringTok{\textquotesingle{}英语 \textquotesingle{}}\NormalTok{),}

\NormalTok{(}\ControlFlowTok{case} \ControlFlowTok{when}\NormalTok{ Chinese}\OperatorTok{\textgreater{}=}\DecValTok{85} \ControlFlowTok{then} \StringTok{\textquotesingle{}优秀 \textquotesingle{}} \ControlFlowTok{when}\NormalTok{ Chinese}\OperatorTok{\textgreater{}=}\DecValTok{60} \ControlFlowTok{then} \StringTok{\textquotesingle{}及格 \textquotesingle{}}\NormalTok{else }\StringTok{\textquotesingle{}不及格 \textquotesingle{}}\NormalTok{end)}\KeywordTok{as} \StringTok{\textquotesingle{}语文 \textquotesingle{}}\NormalTok{)}

\KeywordTok{FROM}\NormalTok{ score;}
\end{Highlighting}
\end{Shaded}

\hypertarget{ux7ea6ux675f}{%
\section{约束}\label{ux7ea6ux675f}}

\hypertarget{ux6982ux8ff0ux548cux5206ux7c7b}{%
\subsection{概述和分类}\label{ux6982ux8ff0ux548cux5206ux7c7b}}

\begin{enumerate}
\def\labelenumi{\arabic{enumi}.}
\tightlist
\item
  概念
\end{enumerate}

约束是\textbf{作用表中字段}上的规则,用于限制存储在表中的数据

\begin{enumerate}
\def\labelenumi{\arabic{enumi}.}
\setcounter{enumi}{1}
\tightlist
\item
  作用
\end{enumerate}

保证数据库中数据的\textbf{正确性、有效性和完整性}

\begin{enumerate}
\def\labelenumi{\arabic{enumi}.}
\setcounter{enumi}{2}
\item
  分类

  \begin{longtable}[]{@{}
    >{\raggedright\arraybackslash}p{(\columnwidth - 4\tabcolsep) * \real{0.1067}}
    >{\raggedright\arraybackslash}p{(\columnwidth - 4\tabcolsep) * \real{0.7467}}
    >{\raggedright\arraybackslash}p{(\columnwidth - 4\tabcolsep) * \real{0.1467}}@{}}
  \toprule\noalign{}
  \begin{minipage}[b]{\linewidth}\raggedright
  约束
  \end{minipage} & \begin{minipage}[b]{\linewidth}\raggedright
  \textbf{描述}
  \end{minipage} & \begin{minipage}[b]{\linewidth}\raggedright
  \textbf{关键字}
  \end{minipage} \\
  \midrule\noalign{}
  \endhead
  \bottomrule\noalign{}
  \endlastfoot
  非空约束 & 限制该字段的数据不能为null & NOT NULL \\
  唯一约束 & 保证该字段的所有数据都是\textbf{唯一} ,\textbf{不重复}的 &
  UNIQUE \\
  主键约束 & 主键是一行数据的唯一标识,要求\textbf{非空且唯一} & PRIMARY
  KEY \\
  默认约束 & 保存数据时,如果未指定该字段的值,则采用默认值 & DEFAULT \\
  检查约束 & 保证字段值满足某一个条件(逻辑表达式+比较运算符) & CHECK \\
  外键约束 & 用来让两张表的数据中之间建立连接,保证数据的一致性和完整 &
  FOREIGN KEY \\
  \end{longtable}

  \begin{center}\rule{0.5\linewidth}{0.5pt}\end{center}
\end{enumerate}

注意:约束是作用在表中字段上的,可以在\textbf{创建表}/\textbf{修改表}的时候添加约束

\hypertarget{ux6848ux4f8bux5c55ux793a}{%
\subsection{\texorpdfstring{\textbf{2.}
\textbf{案例展示}}{2. 案例展示}}\label{ux6848ux4f8bux5c55ux793a}}

\begin{Shaded}
\begin{Highlighting}[]
\KeywordTok{create} \KeywordTok{table} \FunctionTok{user}\NormalTok{(}

\KeywordTok{id} \DataTypeTok{int} \KeywordTok{primary} \KeywordTok{key}\NormalTok{ auto\_increment }\KeywordTok{comment} \StringTok{\textquotesingle{}主键 \textquotesingle{}}\NormalTok{,}

\NormalTok{name }\DataTypeTok{varchar}\NormalTok{(}\DecValTok{10}\NormalTok{) }\KeywordTok{not} \KeywordTok{null} \KeywordTok{comment} \StringTok{\textquotesingle{}名字 \textquotesingle{}}\NormalTok{,}

\NormalTok{age }\DataTypeTok{int} \KeywordTok{check}\NormalTok{ ( age}\OperatorTok{\textgreater{}}\DecValTok{0}\CharTok{\&\&age}\OperatorTok{\textless{}}\DecValTok{120}\NormalTok{ ) }\KeywordTok{comment} \StringTok{\textquotesingle{}年龄 \textquotesingle{}}\NormalTok{,  \# mysql 版本要在8 .0才支持该语句 }

\NormalTok{status }\DataTypeTok{char}\NormalTok{(}\DecValTok{1}\NormalTok{) }\KeywordTok{default} \StringTok{\textquotesingle{}1 \textquotesingle{}}\NormalTok{comment }\StringTok{\textquotesingle{}状态 \textquotesingle{}}\NormalTok{,}

\NormalTok{gander }\DataTypeTok{char}\NormalTok{(}\DecValTok{1}\NormalTok{) comment}\StringTok{\textquotesingle{}性别 \textquotesingle{}}

\NormalTok{)}\KeywordTok{comment} \StringTok{\textquotesingle{}用户表 \textquotesingle{}}\NormalTok{;}
\end{Highlighting}
\end{Shaded}

\hypertarget{ux5916ux952eux7ea6ux675f}{%
\subsection{\texorpdfstring{\textbf{3.}
\textbf{外键约束}}{3. 外键约束}}\label{ux5916ux952eux7ea6ux675f}}

\textbf{3.} \textbf{1} \textbf{语法}

添加外键:保证数据的完整性和一致性

\begin{Shaded}
\begin{Highlighting}[]
\NormalTok{\#建表前添加外键}

\KeywordTok{CREATE} \KeywordTok{TABLE}\NormalTok{ 表名\{}

\NormalTok{     字段名 数据类型 ,}

     \OperatorTok{..}\NormalTok{.}

\NormalTok{     [}\KeywordTok{CONSTRAINT}\NormalTok{][外键名称]}\KeywordTok{FOREIGN} \KeywordTok{KEY}\NormalTok{(外键字段名)}\KeywordTok{REFERENCES}\NormalTok{ 主表(主表列名) \};}

\NormalTok{\#建表后进行添加外键}
\KeywordTok{ALTER} \KeywordTok{TABLE}\NormalTok{ 表名 }\KeywordTok{ADD} \KeywordTok{CONSTRAINT}\NormalTok{ 外键名称 }\KeywordTok{FOREIGN} \KeywordTok{KEY}\NormalTok{ (外键字段名)}\KeywordTok{REFERENCES}\NormalTok{ 主表(主表列 名);\#表名连接主表}

\KeywordTok{ALTER} \KeywordTok{TABLE}\NormalTok{ 表名 }\KeywordTok{ADD} \KeywordTok{CONSTRAINT}\NormalTok{ FK\_表名\_外键字段名 }\KeywordTok{FOREIGN} \KeywordTok{KEY}\NormalTok{ (外键字段名)}\KeywordTok{REFERENCES}\NormalTok{ 主(父)表(主表列名);}
\end{Highlighting}
\end{Shaded}

删除外键

\begin{Shaded}
\begin{Highlighting}[]
\KeywordTok{ALTER} \KeywordTok{TABLE}\NormalTok{ 表名 }\KeywordTok{DROP} \KeywordTok{FOREIGN} \KeywordTok{KEY}\NormalTok{ 外键名称;}
\end{Highlighting}
\end{Shaded}

删除/更新行为(DELETE/UPDATE)

\begin{longtable}[]{@{}
  >{\raggedright\arraybackslash}p{(\columnwidth - 2\tabcolsep) * \real{0.1667}}
  >{\raggedright\arraybackslash}p{(\columnwidth - 2\tabcolsep) * \real{0.8333}}@{}}
\toprule\noalign{}
\begin{minipage}[b]{\linewidth}\raggedright
\textbf{行为}
\end{minipage} & \begin{minipage}[b]{\linewidth}\raggedright
\textbf{说明}
\end{minipage} \\
\midrule\noalign{}
\endhead
\bottomrule\noalign{}
\endlastfoot
NO ACTION &
当父表中删除/更新对应纪录时,首先检查该记录是否有对应外键,如果有则不允许删除/更
新。 \\
RESTRICT & 同上 \\
CASCADE &
当父表中删除/更新对应纪录时,首先检查该记录是否有对应外键,如果有则也删除/更新外键
在字表中的记录 \\
SET NULL &
当父表中删除/更新对应纪录时,首先检查该记录是否有对应外键,如果有则设置字表中该外
键值为null \\
SET DEFAULT &
父表有变更时,字表将外键列设置成一个默认值(Innodb不支持) \\
\end{longtable}

语法:

\begin{Shaded}
\begin{Highlighting}[]
\KeywordTok{ALTER} \KeywordTok{TABLE}\NormalTok{ 表名 }\KeywordTok{ADD} \KeywordTok{CONSTRAINT}\NormalTok{ 外键名称 }\KeywordTok{FOREIGN} \KeywordTok{KEY}\NormalTok{ (外键字段)}\KeywordTok{References}\NormalTok{ 主表名(字表字 段名) }\KeywordTok{ON} \KeywordTok{UPDATE} \KeywordTok{CASCADE} \KeywordTok{ON} \KeywordTok{DELETE} \KeywordTok{CASCADE}\NormalTok{;}
\end{Highlighting}
\end{Shaded}

\hypertarget{ux591aux8868ux67e5ux8be2}{%
\section{多表查询}\label{ux591aux8868ux67e5ux8be2}}

\hypertarget{ux591aux8868ux5173ux7cfb}{%
\subsection{多表关系}\label{ux591aux8868ux5173ux7cfb}}

概述:

由于业务之间相互关联,所以各个表结构之间也存在着各种联系,基本分为三种

\hypertarget{ux4e00ux5bf9ux591aux591aux5bf9ux4e00}{%
\subsubsection{●
一对多(多对一)}\label{ux4e00ux5bf9ux591aux591aux5bf9ux4e00}}

案例:部门与员工的关系

关系:一个部门对应多个员工, 一个员工只能对应一个部门

实现:\textbf{在多的一方建立外键,指向一的一方的主键}(员工表为多表,部门表就为一表)

\hypertarget{ux591aux5bf9ux591a}{%
\subsubsection{\texorpdfstring{\textbf{●
多对多}}{● 多对多}}\label{ux591aux5bf9ux591a}}

案例:学生和课程的关系

关系:一个雪上可以选修多门课程, 一门课程也可以供多个学生选择

实现:\textbf{建立第三张中间表},中间表至少包含两个外键,分别\textbf{关联两方主键}(primary
key)

\hypertarget{ux4e00ux5bf9ux4e00}{%
\subsubsection{\texorpdfstring{\textbf{●
一对一}}{● 一对一}}\label{ux4e00ux5bf9ux4e00}}

案例:用户与用户详情的关系

关系:一对一关系,多用于单表拆分,将一张表的\textbf{基础字段}放在一张表中,
\textbf{其他详情字段放在另一张} \textbf{表中},以提升操作效率

实现:在任意的一方加入外键,关联另外一方的主键,并且设置外键为唯一的(UNIQUE)

\hypertarget{ux591aux8868ux67e5ux8be2-1}{%
\subsection{2.多表查询}\label{ux591aux8868ux67e5ux8be2-1}}

\hspace{0pt} ● 概述: 指从多张表中查询数据

笛卡尔积:笛卡尔乘积是指在数学中个,两个集合A集合和B集合的所有组合情况。(\textbf{多表查询时,需要消除无效的笛卡尔积(连接条件)}
)

\begin{Shaded}
\begin{Highlighting}[]
\NormalTok{\#多表查询}

\KeywordTok{SELECT} \OperatorTok{*} \KeywordTok{FROM}\NormalTok{ 父表 ,外表;}
\end{Highlighting}
\end{Shaded}

\hypertarget{ux591aux8868ux67e5ux8be2ux5206ux7c7b}{%
\subsubsection{\texorpdfstring{\textbf{●
多表查询分类}}{● 多表查询分类}}\label{ux591aux8868ux67e5ux8be2ux5206ux7c7b}}

\hspace{0pt} 连接查询

\begin{enumerate}
\def\labelenumi{\arabic{enumi}.}
\tightlist
\item
  内连接
\end{enumerate}

\hspace{0pt} 相当于查询A、B\textbf{交集}部分数据

\begin{enumerate}
\def\labelenumi{\arabic{enumi}.}
\setcounter{enumi}{1}
\tightlist
\item
  外连接
\end{enumerate}

\hspace{0pt} •左外连接

\hspace{0pt} •右外连接

\begin{enumerate}
\def\labelenumi{\arabic{enumi}.}
\setcounter{enumi}{2}
\tightlist
\item
  自连接
\end{enumerate}

\hspace{0pt} 当前表与自身的连接查询,自连接\textbf{必须使用表别名}

\begin{enumerate}
\def\labelenumi{\arabic{enumi}.}
\setcounter{enumi}{3}
\tightlist
\item
  联合查询
\end{enumerate}

\hspace{0pt} •子查询

\hypertarget{ux8fdeux63a5ux67e5ux8be2}{%
\subsection{3.连接查询}\label{ux8fdeux63a5ux67e5ux8be2}}

\hypertarget{ux5185ux8fdeux63a5}{%
\subsubsection{\texorpdfstring{\textbf{3.1}
\textbf{内连接}}{3.1 内连接}}\label{ux5185ux8fdeux63a5}}

\hspace{0pt} ● 隐式内连接

\begin{Shaded}
\begin{Highlighting}[]
\KeywordTok{SELECT}\NormalTok{ 字段列表 }\KeywordTok{FROM}\NormalTok{ 表1,表2 }\KeywordTok{WHERE}\NormalTok{ 条件 . . .;}

\NormalTok{\#例子 emp员工表 dept部门表}

\KeywordTok{SELECT}\NormalTok{ emp.name,dept.name }\KeywordTok{FROM}\NormalTok{ emp,dept }\KeywordTok{WHERE}\NormalTok{ emp.dept\_id}\OperatorTok{=}\NormalTok{dept.}\KeywordTok{id}\NormalTok{;}
\end{Highlighting}
\end{Shaded}

\hspace{0pt} ● 显式内连接

\begin{Shaded}
\begin{Highlighting}[]
\KeywordTok{SELECT}\NormalTok{ 字段列表 }\KeywordTok{FROM}\NormalTok{ 表1,[}\KeywordTok{INNER}\NormalTok{] }\KeywordTok{JOIN}\NormalTok{ 表2 }\KeywordTok{ON}\NormalTok{ 连接条件;}

\NormalTok{\#例子}

\KeywordTok{SELECT}\NormalTok{ e.name,d.name }\KeywordTok{FROM}\NormalTok{ emp e }\KeywordTok{INNER} \KeywordTok{JOIN}\NormalTok{ dept d }\KeywordTok{ON}\NormalTok{ e.dept\_id}\OperatorTok{=}\NormalTok{d.}\KeywordTok{id}\NormalTok{;}
\end{Highlighting}
\end{Shaded}

\hypertarget{ux5916ux8fdeux63a5}{%
\subsubsection{\texorpdfstring{\textbf{3.2}
\textbf{外连接}}{3.2 外连接}}\label{ux5916ux8fdeux63a5}}

\hspace{0pt} ● 左外连接(LEFT JOIN)

\begin{Shaded}
\begin{Highlighting}[]
\KeywordTok{SELECT}\NormalTok{ 字段列表 }\KeywordTok{FROM}\NormalTok{ 表1 }\KeywordTok{LEFT}\NormalTok{ [}\KeywordTok{OUTER}\NormalTok{] }\KeywordTok{JOIN}\NormalTok{ 表2 }\KeywordTok{ON}\NormalTok{ 条件 \#字段列表根据需求可进行优化 因为按照sql语句的执行属性可知select是在后面执行}
\end{Highlighting}
\end{Shaded}

相当于查询表1(左表)的所有数据和包含表1和表2交集部分的数据

\hspace{0pt} ● 右外连接(RIGHT JOIN)

\begin{Shaded}
\begin{Highlighting}[]
\KeywordTok{SELECT}\NormalTok{ 字段列表 }\KeywordTok{FROM}\NormalTok{ 表1 }\KeywordTok{RIGHT}\NormalTok{ [}\KeywordTok{OUTER}\NormalTok{] }\KeywordTok{JOIN}\NormalTok{ 表2 }\KeywordTok{ON}\NormalTok{ 条件}
\end{Highlighting}
\end{Shaded}

相当于查询表2(右表)的所有数据和包含表1和表2交集部分的数据

\hypertarget{ux81eaux8fdeux63a5}{%
\subsubsection{\texorpdfstring{\textbf{3.3}
\textbf{自连接}}{3.3 自连接}}\label{ux81eaux8fdeux63a5}}

\hspace{0pt} ● 语法

\begin{Shaded}
\begin{Highlighting}[]
\NormalTok{\# 把一张表看做成两张表 通过唯一 的id做为自连接的条件 managerid}\OperatorTok{=}\KeywordTok{id}

\NormalTok{\# 自连接必须给表名取别名! ! !}

\KeywordTok{SELECT}\NormalTok{ 字段列表 }\KeywordTok{FROM}\NormalTok{ 表A 别名A }\KeywordTok{JOIN}\NormalTok{ 表A 别名B }\KeywordTok{ON}\NormalTok{ 条件 . . .;}

\NormalTok{\# 举例:}

\KeywordTok{SELECT}\NormalTok{ a.name,b.name }\KeywordTok{FROM}\NormalTok{ emp a }\KeywordTok{JOIN}\NormalTok{ emp b }\KeywordTok{ON}\NormalTok{ a.managerid}\OperatorTok{=}\NormalTok{b.}\KeywordTok{id}\NormalTok{;}
\end{Highlighting}
\end{Shaded}

\hypertarget{ux8054ux5408ux67e5ux8be2}{%
\subsubsection{\texorpdfstring{\textbf{3.4}
\textbf{联合查询}}{3.4 联合查询}}\label{ux8054ux5408ux67e5ux8be2}}

\hspace{0pt} ● 定义

对于联合查询,就是把多次查询的结果合并起来,形成一个新的查询结果集。

\hspace{0pt} ● 关键词

\begin{Shaded}
\begin{Highlighting}[]
\NormalTok{\# 作为单独一句 在两个sql语句查询中间}

\KeywordTok{UNION}\NormalTok{ \#将查询的结果合并时 ,进行去重}

\KeywordTok{UNION} \KeywordTok{ALL}\NormalTok{ \#直接将查询的结果进行合并}
\end{Highlighting}
\end{Shaded}

注意

\begin{enumerate}
\def\labelenumi{\arabic{enumi}.}
\tightlist
\item
  对于联合查询多表查询的;列数必须保持一致,字段类型也必须保持一致
\item
  union all 会将全部数据直接合并在一起, union 会对合并之后的数据去重
\end{enumerate}

\hypertarget{ux5b50ux67e5ux8be2}{%
\subsection{4.子查询}\label{ux5b50ux67e5ux8be2}}

\hypertarget{ux6982ux8ff0}{%
\subsubsection{\texorpdfstring{\textbf{4.1}
\textbf{概述}}{4.1 概述}}\label{ux6982ux8ff0}}

\hspace{0pt} \textbf{●
概念:}SQL语句中嵌套SELECT语句,成为\textbf{嵌套语句},又称子查询

\begin{Shaded}
\begin{Highlighting}[]
\KeywordTok{SELECT} \OperatorTok{*} \KeywordTok{FROM}\NormalTok{ t1 }\KeywordTok{WHERE}\NormalTok{ column1}\OperatorTok{=}\NormalTok{(}\KeywordTok{SELECT}\NormalTok{ column1 }\KeywordTok{FROM}\NormalTok{ t2);}
\NormalTok{\#子查询外部的语句可以是INSERT}\OperatorTok{/}\KeywordTok{UPDATE}\OperatorTok{/}\KeywordTok{DELETE}\OperatorTok{/}\NormalTok{SELECT的任何一个}
\end{Highlighting}
\end{Shaded}

\hspace{0pt} ● \textbf{查询结果分类}

\begin{enumerate}
\def\labelenumi{\arabic{enumi}.}
\tightlist
\item
  标量子查询(子表查询结果为单个值)
\item
  列子查询(子查询结果为一列)
\item
  行子查询(子查询结果为一行)
\item
  表子查询(子查询结果为多行多列)
\end{enumerate}

\hspace{0pt} ● \textbf{根据子查询位置}

\begin{enumerate}
\def\labelenumi{\arabic{enumi}.}
\tightlist
\item
  WHERE之后
\item
  FROM之后
\item
  SELECT之后
\end{enumerate}

\hypertarget{ux67e5ux8be2ux7ed3ux679cux5206ux7c7b}{%
\subsubsection{\texorpdfstring{\textbf{4.2}
\textbf{查询结果分类}}{4.2 查询结果分类}}\label{ux67e5ux8be2ux7ed3ux679cux5206ux7c7b}}

\hypertarget{ux6807ux91cfux5b50ux67e5ux8be2}{%
\paragraph{\texorpdfstring{\textbf{●
标量子查询}}{● 标量子查询}}\label{ux6807ux91cfux5b50ux67e5ux8be2}}

\begin{enumerate}
\def\labelenumi{\arabic{enumi}.}
\tightlist
\item
  定义:
  子查询返回的结果是单个值(数字,字符串,日期等),最简单的形式,这种子查询称为标量子查询
\item
  常用操作符:= 、\textless\textgreater、\textgreater{} 、\textgreater=、
  \textless{} 、\textless=
\item
  案例展示
\end{enumerate}

\begin{Shaded}
\begin{Highlighting}[]
\NormalTok{\# emp员工信息表}

\NormalTok{\# dept部门表}

\NormalTok{\# 要求查询具体部门对应的员工信息 前提员工表和部门表是已经建立了连接(}\KeywordTok{foreign} \KeywordTok{key}\NormalTok{) \# 分析 首先得知道具体部门的部门ID 进而查询该部门的员工信息}

\NormalTok{\# 可以两种sql写法}

\KeywordTok{SELECT} \KeywordTok{id} \KeywordTok{FROM}\NormalTok{ dept }\KeywordTok{WHERE}\NormalTok{ name}\OperatorTok{=}\StringTok{\textquotesingle{}具体的部门名 \textquotesingle{}}\NormalTok{;}

\KeywordTok{SELECT} \OperatorTok{*} \KeywordTok{FROM}\NormalTok{ emp }\KeywordTok{WHERE}\NormalTok{ dept\_id }\OperatorTok{=}\OtherTok{"上条语句的结果"}\NormalTok{;}

\NormalTok{\# 所以可以整合成一句 形成嵌套}

\KeywordTok{SELECT} \OperatorTok{*} \KeywordTok{FROM}\NormalTok{ emp }\KeywordTok{WHERE}\NormalTok{ dept\_id}\OperatorTok{=}\NormalTok{(}\KeywordTok{SELECT} \KeywordTok{id} \KeywordTok{FROM}\NormalTok{ dept }\KeywordTok{WHERE}\NormalTok{ name}\OperatorTok{=}\StringTok{\textquotesingle{}具体的部门名 \textquotesingle{}}\NormalTok{);}
\end{Highlighting}
\end{Shaded}

\hypertarget{ux5217ux5b50ux67e5ux8be2}{%
\paragraph{\texorpdfstring{\textbf{●
列子查询}}{● 列子查询}}\label{ux5217ux5b50ux67e5ux8be2}}

\begin{enumerate}
\def\labelenumi{\arabic{enumi}.}
\item
  子查询返回的结果是一列(可以是多行 ),这种子查询为列子查询
\item
  常用操作符:IN、NOT IN、ANY、SOME、ALL
\end{enumerate}

\begin{longtable}[]{@{}ll@{}}
\toprule\noalign{}
\textbf{操作符} & \textbf{描述} \\
\midrule\noalign{}
\endhead
\bottomrule\noalign{}
\endlastfoot
IN & 在指定的集合范围之内,多选一 \\
NOT IN & 不在指定的集合范围之内 \\
\textbf{ANY} & 子查询返回列表中,有\textbf{任意一个满足}即可 \\
\textbf{SOME} & 与ANY\textbf{等同},使用SOME的地方都可以使用ANY \\
\textbf{ALL} & 子查询返回列表的\textbf{所有值都必须满足} \\
\end{longtable}

\begin{enumerate}
\def\labelenumi{\arabic{enumi}.}
\setcounter{enumi}{2}
\tightlist
\item
  案例演示
\end{enumerate}

\begin{Shaded}
\begin{Highlighting}[]
\NormalTok{\#查询比财务部所有人工资都高的员工信息}

\NormalTok{\#将获取信息的问题拆分化 在通过嵌套进行优化 形成一句sql语句}

\NormalTok{\# 获取财务部的部门id}

\KeywordTok{SELECT} \KeywordTok{id} \KeywordTok{FROM}\NormalTok{ dept }\KeywordTok{WHERE}\NormalTok{ name }\OperatorTok{=}\StringTok{\textquotesingle{}财务部\textquotesingle{}}\NormalTok{;}

\NormalTok{\# 获取财务中所有人的工资}

\KeywordTok{SELECT}\NormalTok{ salary }\KeywordTok{FROM}\NormalTok{ emp }\KeywordTok{WHERE} \KeywordTok{id}\OperatorTok{=}\NormalTok{(}\KeywordTok{SELECT} \KeywordTok{id} \KeywordTok{FROM}\NormalTok{ dept }\KeywordTok{WHERE}\NormalTok{ name }\OperatorTok{=}\StringTok{\textquotesingle{}财务部\textquotesingle{}}\NormalTok{);}

\NormalTok{\# 获取比财务部所有员工工资都高的员工信息}

\KeywordTok{SELECT} \OperatorTok{*} \KeywordTok{FROM}\NormalTok{ emp }\KeywordTok{WHERE}\NormalTok{ salary}\OperatorTok{\textgreater{}}\KeywordTok{ALL} \KeywordTok{SELECT}\NormalTok{ salary }\KeywordTok{FROM}\NormalTok{ emp }\KeywordTok{WHERE} \KeywordTok{id}\OperatorTok{=}\NormalTok{(}\KeywordTok{SELECT} \KeywordTok{id} \KeywordTok{FROM}\NormalTok{ dept }\KeywordTok{WHERE}\NormalTok{ name }\OperatorTok{=}\StringTok{\textquotesingle{}财务部\textquotesingle{}}\NormalTok{);}
\end{Highlighting}
\end{Shaded}

\hypertarget{ux884cux5b50ux67e5ux8be2}{%
\paragraph{● 行子查询}\label{ux884cux5b50ux67e5ux8be2}}

1 . 子查询返回的结果是一行(可以是多列),这种子查询为行子查询

2 . 常用操作符:IN、 NOT IN、ANY、 SOME、ALL(同上)

3 . 案例演示

\begin{Shaded}
\begin{Highlighting}[]
\NormalTok{\# 查询张无忌的薪资及直属领导相同的员工信息}

\KeywordTok{SELECT}\NormalTok{ salary managerid }\KeywordTok{FROM}\NormalTok{ emp }\KeywordTok{WHERE}\NormalTok{ name}\OperatorTok{=}\OtherTok{"张无忌 "}\NormalTok{;}

\KeywordTok{SELECT} \OperatorTok{*} \KeywordTok{FROM}\NormalTok{ emp }\KeywordTok{WHERE}\NormalTok{ (salary,managerid)}\OperatorTok{=}\NormalTok{(}\KeywordTok{SELECT}\NormalTok{ salary managerid }\KeywordTok{FROM}\NormalTok{ emp }\KeywordTok{WHERE}\NormalTok{ name}\OperatorTok{=}\OtherTok{"张无忌 "}\NormalTok{);}

\NormalTok{\#返回的结果是一行多列 对应的是salary和managerID}
\end{Highlighting}
\end{Shaded}

\hypertarget{ux8868ux5b50ux67e5ux8be2}{%
\paragraph{● 表子查询}\label{ux8868ux5b50ux67e5ux8be2}}

\begin{enumerate}
\def\labelenumi{\arabic{enumi}.}
\tightlist
\item
  表子查询返回的结果是多行多列,这种子查询为列表查询
\item
  常用操作符: IN
\item
  案例演示
\end{enumerate}

\begin{Shaded}
\begin{Highlighting}[]
\NormalTok{\#查询入职日期是 “}\DecValTok{2006}\OperatorTok{{-}}\DecValTok{01}\OperatorTok{{-}}\NormalTok{01”之后的员工信息和部门信息}

\KeywordTok{SELECT} \OperatorTok{*} \KeywordTok{FROM}\NormalTok{ emp }\KeywordTok{where}\NormalTok{ entrydate}\OperatorTok{\textgreater{}}\StringTok{\textquotesingle{}2006{-}01{-}01\textquotesingle{}}\NormalTok{;}

\KeywordTok{SELECT}\NormalTok{ e.}\OperatorTok{*}\NormalTok{,d.}\OperatorTok{*} \KeywordTok{FROM}\NormalTok{ (}\KeywordTok{SELECT} \OperatorTok{*} \KeywordTok{FROM}\NormalTok{ emp }\KeywordTok{where}\NormalTok{ entrydate}\OperatorTok{\textgreater{}}\StringTok{\textquotesingle{}2006{-}01{-}01 \textquotesingle{}}\NormalTok{) e }\KeywordTok{LEFT} \KeywordTok{JOIN}\NormalTok{ dept d }\KeywordTok{on}\NormalTok{ e.dept\_id}\OperatorTok{=}\NormalTok{d.}\KeywordTok{id}\NormalTok{;}
\end{Highlighting}
\end{Shaded}




\end{document}
